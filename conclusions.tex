\section{Conclusions} 
\label{sec:conclusions}
In this paper, we design a novel architecture to efficiently map the huge number of reads onto the human reference genome.
Our proposed architecture focuses on accelerating the compute-intensive Smith-Waterman kernel used by the state-of-the-art aligner, BWA-MEM.
To map the enormous number of seeds generated from the reads, we design an array-based architecture equipped with a large number of simple processing units.
Since the seeds have various sizes, the PE array architecture can utilize the computation resource efficiently compared to the conventional wavefront-based architecture.
Also, our two-level hierarchical architecture can reduce the FPGA fabric utilization for synthesizing data prefetching logics and provide efficient PE allocation.
Furthermore, our design can be integrated to BWA-MEM, which extends the conventional S-W algorithm with pruning techiniques. 
Our FPGA implementation demonstrates about 26.4x speedup compared to a 24-thread Intel Xeon server. 
We also show that our design outperforms wavefront-based implementations by up to 6x under the same FPGA resource.
