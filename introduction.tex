\section{Introduction} 
\label{sec:introduction}

The advent of next-generation sequencing (NGS) technologies dramatically reduces the cost of genome sequencing. 
Today's sequencing technologies can obtain a genome for an individual for \$1,000 or less \cite{Mardis2006}. 
The technology can be widely used in research and is transitioning into the clinic for applications such as precision medicine for cancer treatment. 
Next-generation sequencers are more cost-effective because they generate the sequence 
from very small fragments (reads) of length in the range of a few hundred nucleotides. 
Mapping billions of sequenced fragments onto a genome sequence by taking advantage of the known human genome sequence 
is called resequencing, and it entails tremendous computational challenges.

Current state-of-the-art tools, such as BWA-MEM \cite{BWA}\cite{BWA-SW}\cite{BWA-MEM} and Bowtie 2 \cite{Bowtie}\cite{Bowtie2}, 
carry out the read mapping with two steps. 
First, a seeding step can generate a number of smaller seeds from each read. 
State-of-the-art tools, such as BWA-MEM \cite{BWA-MEM} and Bowtie 2 \cite{Bowtie2}, 
use the backward search to exactly map the seeds to the reference genome encoded by the Burrows-Wheeler Transform (BWT) \cite{BWT}.
%use the backward search \cite{Ferragina2000} to exactly map the seeds to the reference genome encoded by Burrows-Wheeler Transform (BWT) \cite{BWT}.
The mapping process takes only O($m$) time for a seed of length $m$ and is independent of the size of the reference genome.

However, due to the genome differences between individuals and the errors arising from sequencers,
inexact mappings are needed in order to find mappings with gaps (insertions, deletions, and mismatches) allowed. 
In the second step, the classic S-W algorithm is used to find inexact mappings.
Compared to the exact mapping with linear time complexity, the S-W algorithm is a dynamic programming algorithm 
with quadratic complexity; it thus becomes the main computation bottleneck in current state-of-the-art tools like BWA-MEM \cite{BWA-MEM}. 
In this work, we focus on accelerator design for accelerating the S-W algorithm.

The hardware acceleration of the S-W algorithm has been studied in multi-core CPUs with SIMD vector engines, GPUs, and FPGAs. 
In \cite{Wozniak1997}\cite{Farrar2007}\cite{Rognes2011}, 
the authors redesigned the S-W algorithms based on the architecture of a multi-core CPU with SIMD vector engines for acceleration. 
The wavefront-based \cite{Wozniak1997} and column-based \cite{Farrar2007}\cite{Rognes2011} algorithms were proposed to map to the vector engines, respectively.
Several GPU implementations, such as the work in \cite{Manavski2008} and CUDASW++ 3.0 \cite{Liu2013}, demonstrated the speedups over SIMD CPU implementations \cite{Farrar2007}\cite{Rognes2011}.
The research in \cite{Preusser2012}\cite{RaceLogic}\cite{Zhang2007}\cite{Yu2005}\cite{Kim2011} exploited the anti-diagonal parallelism of S-W algorithm 
and implemented wavefront-based architectures on FPGA.
However, none of the proposed methods can be effectively integrated into BWA-MEM where the S-W algorithm is extended
to prune the solution space for better performance of long reads.

The extended S-W algorithm used in BWA-MEM introduces three major challenges, which are not considered in conventional S-W hardware acceleration techniques \cite{Wozniak1997}\cite{Farrar2007}\cite{Rognes2011}\cite{Manavski2008}\cite{Liu2013}\cite{Preusser2012}\cite{RaceLogic}\cite{Zhang2007}\cite{Yu2005}\cite{Kim2011}. 
First, a huge number of reads at the billion scale need to be processed with high throughput.
Second, in BWA-MEM, the seeding process further generates more seeds with highly varied sizes to be processed by the S-W algorithm. 
Conventional wavefront-based architectures cannot utilize computational resources efficiently when the input sizes vary. 
Third, the techniques developed for conventional SIMD algorithms \cite{Wozniak1997}\cite{Farrar2007}\cite{Rognes2011} 
and GPU/FPGA-based architectures \cite{Manavski2008}\cite{Liu2013}\cite{Preusser2012}\cite{RaceLogic}\cite{Zhang2007}\cite{Yu2005}\cite{Kim2011}, 
cannot be directly adopted due to the pruning heuristic used in BWA-MEM.

In this paper, we propose a novel architecture that can be mapped into the FPGA.
Our architecture novelties can be summarized as follows.
(1)	We propose an array-based architecture for processing the enormous number of seeds.
Each processing element (PE) in an PE array is a lightweight computation unit which is flexible enough to support the pruning S-W algorithm 
but much more resource-efficient than the expensive wavefront-based design. 
With the simple PE design, we can maximize the number of PEs on the FPGA board 
and thus process the enormous number of seeds in a high-throughput fashion.
Furthermore, this design style can adapt to seeds with various sizes better, 
which achieves a higher utilization rate of PEs than that of wavefront-based architectures.
(2)	We provide a two-level hierarchical architecture, which is composed of multiple PE arrays, for resource management.
Each PE array has a task distributor for PE allocation and uses a single AXI interface to prefetch data from off-chip DRAM.
Instead of using one AXI interface per PE, this design strategy can significantly save FPGA fabrics while satisfying the off-chip bandwidth demand.
(3) Our design supports the pruning technique, and thus can be directly integrated with the state-of-the-art BWA-MEM \cite{BWA-MEM}. 
The pruning technique can reduce the solution space of the S-W algorithm significantly, 
especially when the length a read grows longer in the near future \cite{BWA-SW}\cite{BWA-MEM}.

Our FPGA implementation demonstrates a 26.4x speedup compared to a 24-thread Intel Haswell Xeon server for the extended Smith-Waterman algorithm used in the state-of-the-art aligner, BWA-MEM.
We also show that our design outperforms wavefront-based implementations by up to 6x with the same FPGA resource utilization.
