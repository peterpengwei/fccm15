\section{Introduction} 
\label{sec:introduction}

Next-generation sequencing (NGS) technologies have attracted a large amount of attentions from both researchers and clinicians. 
Human DNA samples are chopped into billions of small fragments, called reads, and a sequencer determines the order of each read's nucleotides. 
A software aligner then maps all the billions of sequenced reads onto a reference human genome, entailing tremendous computational challenges.
%%%The size of a DNA sequence is often measured in base pairs (bps). 
A read typically consists of hundreds of nucleotides or base pairs (bps), while the reference genome contains 3.2 billion bps \cite{Mardis2008}.
%%The advent of next-generation sequencing (NGS) technologies dramatically reduces the cost of genome sequencing. 
%%Today's sequencing technologies can obtain a genome for an individual for \$1,000 or less \cite{Mardis2006}. 
%%The technology can be widely used in research and is transitioning into the clinic for applications such as precision medicine for cancer treatment. 
%%Next-generation sequencers are more cost-effective because they generate the sequence 
%%from very small fragments (reads) of length in the range of a few hundred nucleotides. 
%%Mapping billions of sequenced fragments onto a genome sequence by taking advantage of the known human genome sequence 
%%is called resequencing, and it entails tremendous computational challenges.

State-of-the-art read aligners, such as BWA-MEM \cite{BWA-MEM} and Bowtie 2 \cite{Bowtie}\cite{Bowtie2}, 
carry out the read mapping with two steps. 
First, each read is fragmented into small pieces called seeds, and mapped to the reference genome. 
The mapping of seeds to the reference genome is required to be exact, i.e. no gap or mismatch is allowed. 
A backward search algorithm based on the Burrows-Wheeler Transform (BWT) \cite{BWT} takes only O($m$) time for a seed of length $m$ mapping to the super-long reference genome, independent of the size of the reference genome, and therefore employed by almost all contemporary sequence aligners.

In the second step, each seed map gets extended leftward and rightward till the entire read. 
The extensions are deemed as inexact mappings, in which gaps and mismatches are allowed. A pre-defined scoring function is provided for evaluating the effectiveness of inexact mappings. Only the ones achieving high enough scores are recorded in the output. 
The S-W algorithm, a dynamic programming algorithm with quadratic time complexity, is the classic approach to address this problem. 
It then becomes the main computation bottleneck in state-of-the-art tools like BWA-MEM \cite{BWA-MEM}, 
compared to the BWT-based mapping with only linear time complexity. 
In this work, we focus on accelerating the S-W algorithm in BWA-MEM. The methodology, however, can be applied to other contemporary aligners, such as Bowtie 2 and LAST. 

The S-W algorithm is inherently anti-diagonal parallelizable, since the elements along each anti-diagonal in the S-W matrix are independent of each other, and each element only depends on three elements from the previous two anti-diagonals \cite{Edmiston1988}. 
This feature has been explored in different ways on different platforms \cite{Preusser2012}\cite{RaceLogic}\cite{Zhang2007}\cite{Kim2011}\cite{Lam2013}. 
It works fairly well when inputs are homogeneous. 
However, the solutions targeting at the general S-W algorithm do not fit very well for the customized version used in BWA-MEM. 
First, a huge number of reads at the billion scale need to be processed with high throughput. 
Conventional approaches may result in an overemphasis on inner-task parallelism. 
Second, conventional wavefront-based architectures cannot utilize computational resources efficiently when the input sizes vary sharply, which is the very case in BWA-MEM. 
Third, the pruning heuristic used in BWA-MEM prevents conventional solutions from being adopted directly and efficiently.

In this paper, we propose an architecture to address these issues.
Our architecture novelties can be summarized as follows. 
(1) We propose an array-based architecture for processing the enormous number of reads in a high-throughput fashion as well as adapting to inputs with widely varied sizes better. 
(2) We provide a two-level hierarchical architecture for resource management, saving FPGA fabrics while satisfying the off-chip bandwidth demand.
(3) Our design supports the pruning technique, reducing the solution space of S-W significantly. 

Our FPGA implementation demonstrates a 26.4x speedup compared to a 24-thread Intel Haswell Xeon server for S-W in BWA-MEM. 
We also show that our design outperforms wavefront-based implementations by up to 6x with the same FPGA resource utilization.
